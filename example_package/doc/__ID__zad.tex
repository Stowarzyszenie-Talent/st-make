\documentclass[zad]{talentTex}

\title{Przykładowy tytuł zadania}
\id{__ID__}
\contest{SKN 2023/24}
\day{}
\round{17}
\group{A}
\Memory[MB]{256}
\Time[s]{2,4}

\start

Lorem ipsum dolor ssist amdet, consectetur adipiscing elit.
Donec rhoncus torstor vitae lorem luctus maximus.
Donec bibendum, dui eget diganissim viverra, lectus leo scelerisque metus, ut sagittis ligula nisl ut purus.
Etiam imperdiet luctus leo, et molestie diam congue auctor.
Sed sapien eros, imperdiet in fermentum \tc{a}, faucibus ut ex cośtam cośtam.

Lorem ipsum dolor sit amet, consectetur adipiscing elit.
Donec rhoncus tortor vitae lorem luctus maximus.

\tSection{Wejście}

W pierwszym wierszu standardowego wejścia znajduje się jedna liczba \tc{n} ($1 \leq \tc{n} \leq 10^6$), oznaczająca parametr z~wejścia.
W drugim wierszu znajduje się \tc{n} liczb $\tc{a_i}$ $(1 \leq \tc{n} \leq 10^6)$.

\tSection{Wyjście}

Lorem ipsum dolor ssist amdet, consectetur adipiscing elit.
Donec rhoncus torstor vitae lorem luctus maximus.
Donec bibendum, dui eget diganissim viverra, lectus leo scelerisque metus, ut sagittis ligula nisl ut purus.

W pierwszym wierszu wyjścia powinna znajdować się jedna liczba, wynik z zadania.

\tSection{Przykład}

\example[v]{a}
\example[h]{a}
\example{a}

\tSection{Wyjaśnienie przykładów}

\tSmallSection{Przykład 1}

Lorem ipsum dolor ssist amdet, consectetur adipiscing elit.
Donec rhoncus torstor vitae lorem luctus maximus.
Donec bibendum, dui eget diganissim viverra, lectus leo scelerisque metus, ut sagittis ligula nisl ut purus.

\tSmallSection{Przykład 2}

Lorem ipsum dolor ssist amdet, consectetur adipiscing elit.

\tSection{Testy ocen}

\ocen{
  \testOcen{1ocen}{$\tc{n} = 69$, Założenie jakieś}
  \testOcen{0b}{$\tc{n} = 420$, założenie jakieś 2}
}

\tSection{Ocenianie}

Zestaw testów dzieli się na następujące podzadania.
Testy do każdego podzadania składają się z jednej lub większej liczby osobnych grup testów.

\subtaskTable{
  \subtask{30}{$\tc{n} \leq 100$}
  \subtask{20}{$\tc{n} \leq 1000$  Dwie\\ Linie}
  \subtask{50}{Bez ograniczeń}
}

\finish

%%%%%%%%%%%%%%%%%%
% Przydatne komendy:
% \pagebreak % komenda zaczynająca nową stronę pdfa
% ~ % słowa oddzielone '~' zamiast ' ' w~taki sposób będą zawsze koło siebie
% pusta linijka (lub \par) rozpoczyna nowy akapit.
% \footnote{text} tworzy adnotację na dole strony do miejsca w którym użyto.
% \color{color}{text} koloruje tekst
% \vspace{5pt} tworzy odstęp.
% $text_mat$ % pozwala na pisanie wyrażeń matematycznych
% Dokumentacja:
%   https://www.overleaf.com/learn

%%%%%%%%%%%%%%%%%%
% Komendy talentowe:
% \start  % Rozpoczyna treść, musi być na samym początku treści zadania i po deklaracjach.
% \finish % Kończy treść, musi być na samym końcu treści zadania.
% \tSection{text} % Nagłówek w stylu talentu.
% \tSmallSection{text} % Mały nagłówek w stylu talentu.
% \tc{text} % Styl używany do oznaczania zmiennych.
% \example[h/v]{id} % Wstawia test przykłądowy "abc0id" z paczki.
%   abc0.in -> \example{}, abc0xy.in -> \example{xy}, abc0x.in -> \example[v]{x}.
%   Opcjonalnie można dodać położenie testów, 'h' - horyzontalnie, 'v' - pionowo, domyślną wartością jest h.  
%   Przy kompilacji testy są automatycznie czytane z folderów ./in i ./out, więc upewnij się że się tam znajdują.
% \ocen{ \testOcen{}{}... } % Lista wszystkich testów ocen, podajemy do niej komendy \testOcen{}{}.
% \testOcen{test}{text} % Pojedyńczy test ocen z opisem. Podajesz nazwe testu (1ocen, 0c, 2ocen) i opis
% \subtaskTable[point]{ \subtask{}{}... } % Tworzy tabelę z podzadaniami, podajemy do niej komendy \subtask{}{}.
%   Sprawdzi czy punkty sumują się do podanej opcjonalnie ilości punktów (domyślnie 100) a jak nie to spowoduje błąd kompilacji z podaną przyczyną.
% \subtask{point}{text} % Tworzy pojedyńczy wiersz tabeli opisujący podzadanie, o danej ilości punktów i z danym ograniczeniem.
% \twocol{}{} % \twocol[szerokość1][przerwa][t/b/c]{kolumna1}{kolumna2} % Tworzy 2 kolumny z podaną zawartością.
%   Opcjonalnie można podać ułamek szerokości pierwszej kolumny (domyślnie 0.5), ułamek szerokości przerwy ( domyślnie0).
%   Szerokość 2 kolumny dopełni się do całości. Oraz opcjonalnie wybrać łączenie lini bazowej (domyślnie t). 
% \imgt[szerokość]{plik}{opis} % Wstawia zdjęcie z opisem u góry. Można opcjonalnie zmienić szerokość zdjęcia (0.8).
% \imgb[szerokość]{plik}{opis} % Wstawia zdjęcie z opisem na dole. Można opcjonalnie zmienić szerokość zdjęcia (0.8).
% \img{plik} % \img[szerokość][opis][t/b]{plik} % Wstawia zdjęcie. Można opcjonalnie zmienić szerokość zdjęcia (0.8).
%   Pozostałe 2 parametry tworzą opis jak polecenia \imgb i \imgt gdzie t i b to góra lub dół.
% \title{} \id{} % Ustawiają tytuł i id. Są obowiązkowe.
% \contest{} % Informuje jaki nazywa się kontest i jest wyświetlany w nagłówku.
% \day{} \round{} \group{} % Wyświetla się w nagłówku. Automatycznie dodaje przed nazwe (np \day{2} -> Dzień: 2)
% \Memory[]{} \Time[]{} % Też się wyświetla z nazwą. Dodatkowo dodaje jednostkę którą opcjonalnie można zmienić.

\documentclass[zad]{talentTex}

\title{Przykładowy tytuł zadania}
\id{abc}
\iomode{stdin} % stdin
\Memory{256} % w MB
\Time{2,4} % w sekundach
\group{A} 
\konkurs{SKN 2023/24}
\day{Runda 17}

%%%%%%%%%%%%%%%%%%
% Przydatne komendy:
% \pagebreak % komenda zaczynająca nową stronę pdfa
% ~ % słowa oddzielone '~' zamiast ' ' w~taki sposób będą zawsze koło siebie
% $text_mat$ % pozwala na pisanie wyrażeń matematycznych
% Dokumentacja:
%   https://www.overleaf.com/learn

%%%%%%%%%%%%%%%%%%
% Komendy talentowe
% \start{}  % Rozpoczyna treść, musi być na samym początku treści zadania.
% \finish{} % Kończy treść, musi być na samym końcu treści zadania.
% \tSection{text} % Nagłówek w stylu talentu.
% \tCustomSection{text}{xpt} % Nagłówek w stylu talentu, z możliwością ustawienia odstępu 'x' od poprzedniego akapitu.
% \tSmallSection{text} % Mały nagłówek w stylu talentu.
% \tc{text} % Styl używany do oznaczania zmiennych.
% \makecompactexample{id}   % 2 style dodawania automatycznie testów "zad0{id}" z paczki.
% \makestandardexample{id}  % compact - obok siebie, standard - pod sobą.
%     % Przy kompilacji testy są automatycznie czytane z folderów ./in i ./out
% \ocen{text} % Lista wszystkich testów ocen.
% \testOcen{text}{text2} % Pojedyńczy test ocen z opisem.
% \ocenTable{text} % Tabela z podzadaniami.
% \ocenRow{text} % Pojedyńczy wiersz tabeli: kolejne komórki powinne być rodzielone znakiem &.
% \ocenElement{text} % Jeśli chcesz mieć 2 linie w pojedyńczej komórce tabeli.

\start{}

Lorem ipsum dolor ssist amdet, consectetur adipiscing elit.
Donec rhoncus torstor vitae lorem luctus maximus.
Donec bibendum, dui eget diganissim viverra, lectus leo scelerisque metus, ut sagittis ligula nisl ut purus.
Etiam imperdiet luctus leo, et molestie diam congue auctor. Sed sapien eros, imperdiet in fermentum $\tc{a}$, faucibus ut ex cośtam. 
% pusta linijka, by zacząć akapit od wcięcia

Lorem ipsum dolor sit amet, consectetur adipiscing elit.
Donec rhoncus tortor vitae lorem luctus maximus.

\tSection{Wejście}

W pierwszym wierszu standardowego wejścia znajduje się jedna liczba $\tc{n}$ ($1\leq \tc{n}\leq 1\ 000\ 000$), oznaczająca parametr z~wejścia.

\tSection{Wyjście}

Lorem ipsum dolor ssist amdet, consectetur adipiscing elit.
Donec rhoncus torstor vitae lorem luctus maximus.
Donec bibendum, dui eget diganissim viverra, lectus leo scelerisque metus, ut sagittis ligula nisl ut purus.

W pierwszym wierszu wyjścia powinna znajdować się jedna liczba $\tc{z}$, wynik z zadania.

\tSection{Przykład}

\makestandardexample{a}  % abc0.in -> {}, abc0x.in -> {x}
\makecompactexample{a}  % testy muszą być wygenerowane więc użyj st-make ingen outgen

\tCustomSection{Wyjaśnienie przykładów}{5pt}

\tSmallSection{Przykład 1}

Lorem ipsum dolor ssist amdet, consectetur adipiscing elit.
Lorem ipsum dolor ssist amdet, consectetur adipiscing elit.
Donec rhoncus torstor vitae lorem luctus maximus.
Donec bibendum, dui eget diganissim viverra, lectus leo scelerisque metus, ut sagittis ligula nisl ut purus.

\tSmallSection{Przykład 2}

Lorem ipsum dolor ssist amdet, consectetur adipiscing elit.
Lorem ipsum dolor ssist amdet, consectetur adipiscing elit.


\tSection{Testy ocen:}
\ocen{
  \testOcen{1ocen}{$\tc{n} = 69$, Założenie jakieś}
  \testOcen{2ocen}{$\tc{n} = 420$, założenie jakieś 2}
}

\tSection{Ocenianie}

Zestaw testów dzieli się na następujące podzadania.
Testy do każdego podzadania składają się z jednej lub większej liczby osobnych grup testów.

% Alternatywnie \customOcenTable{|c|c|c|}{\tc{Nr} & \tc{Ograniczenia} & \tc{Punkty}}{reszta tabeli}

\ocenTable{
  \ocenRow{1 & $\tc{n} \leq 100$ & 30}
  \ocenRow{2 & \ocenElement{$\tc{n} \leq 1000$  Dwie\\ Linie} & 20}
  \ocenRow{3 & Bez ograniczeń & 50}
}

\finish{}

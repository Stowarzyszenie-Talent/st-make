\documentclass[zad]{talentTex}
\title{Na ile części podzielić?}
\id{abc}
\iomode{stdin} % stdin / interactive
\Time{10,5} % Nie dodawać "s"
\Memory{256} % Nie dodawać "MB"
% Poniższe pola uzupełnia kierownik konkursu.
\konkurs{SKN 2022/23}
\day{Dzień 1}
\group{A} % Sama literka

%%%%%%%%%%%%%%%%%%
% Przydatne komendy:
% \pagebreak % zacznij nową stronę 
% $text$ % pozwala na pisanie wyrażeń matematycznych
% Dokumentacja:
%   https://www.overleaf.com/learn

%%%%%%%%%%%%%%%%%%
% Komendy talentowe
% \start{}  % Początek i koniec treści zadania
% \finish{} % Cała treść powinna być między tymi znacznikami
% \tSection{text} % Nagłówek w stylu talentu
% \tNoSkipSection{text} % Jak wyżej, tylko bez odstępu między akapitami
% \tSmallSection{text} % Mały nagłówek w stylu talentu
% \tc{text} % Styl używany do oznaczania zmiennych
% \makecompactexample{id}   % 2 style dodawania testów "zad0{id}" z paczki
% \makestandardexample{id}
%     % Przy kompilacji testy są automatycznie czytane z folderów ./in i ./out
% \ocen{text} % Lista wszystkich testów ocen
% \testOcen{text}{text2} % Pojedyńczy test ocen z opisem
% \ocenTable{text} % Tabela z podzadaniami
% \ocenRow{text} % Pojedyńczy wiersz tabeli: kolejne komórki powinne być rodzielone znakiem &
% \ocenElement{text} % Jeśli chcesz mieć 2 linie w pojedyńczej komórce tabeli

\start{}

Lorem ipsum dolor ssist amdet, consectetur adipiscing elit.
Donec rhoncus torstor vitae lorem luctus maximus.
Donec bibendum, dui eget diganissim viverra, lectus leo scelerisque metus, ut sagittis ligula nisl ut purus.
Etiam imperdiet luctus leo, et molestie diam congue auctor. Sed sapien eros, imperdiet in fermentum $\tc{a}$, faucibus ut ex cośtam. 
% pusta linijka, by zacząć akapit od wcięcia

Lorem ipsum dolor sit amet, consectetur adipiscing elit.
Donec rhoncus tortor vitae lorem luctus maximus.

\tSection{Wejście}

W pierwszym i jedynym wierszu wejścia znajduje się jedna liczba $\tc{n}$, ($1\leq \tc{n}\leq 1\ 000\ 000$)

\tSection{Wyjście}

Lorem ipsum dolor ssist amdet, consectetur adipiscing elit.
Donec rhoncus torstor vitae lorem luctus maximus.
Donec bibendum, dui eget diganissim viverra, lectus leo scelerisque metus, ut sagittis ligula nisl ut purus.

W pierwszym i jedynym wierszu wyjścia powinna znajdować się jedna liczba $\tc{z}$, wynik zadania.

\tSection{Przykład}

\makecompactexample{a}  % abc0.in -> \makecompactexample{}
                        % abc0a.in -> \makecompactexample{a}
                        % abc0b.in -> \makecompactexample{b}

\tNoSkipSection{Wyjaśnienie przykładów}{0pt}

\tSmallSection{Przykład 1}

Lorem ipsum dolor ssist amdet, consectetur adipiscing elit.
Lorem ipsum dolor ssist amdet, consectetur adipiscing elit.
Donec rhoncus torstor vitae lorem luctus maximus.
Donec bibendum, dui eget diganissim viverra, lectus leo scelerisque metus, ut sagittis ligula nisl ut purus.

\tSmallSection{Przykład 2}

Lorem ipsum dolor ssist amdet, consectetur adipiscing elit.
Lorem ipsum dolor ssist amdet, consectetur adipiscing elit.


\tSection{Testy ocen:}
\ocen{
  \testOcen{1ocen}{$\tc{n} = 69$, Założenie}
  \testOcen{2ocen}{$\tc{n} = 420$, założenie 2}
}

\tSection{Ocenianie}

Zestaw testów dzieli się na następujące podzadania.
Testy do każdego podzadania składają się z jednej lub większej liczby osobnych grup testów.

% Alternatywnie \customOcenTable{|c|c|c|}{\tc{Nr} & \tc{Ograniczenia} & \tc{Punkty}}{reszta tabeli}

\ocenTable{
  \ocenRow{1 & \ocenElement{$\tc{n} \leq 1000$  Dwie\\ Linie} & 30}
  \ocenRow{2 & $\tc{n} \leq 1\,000\,000$ & 70}
}

\finish{}

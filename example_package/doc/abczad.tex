\documentclass[zad,zawodnik]{sinol}
\title{Tytuł}
\id{abc}
\signature{abc0001}
\pagestyle{fancy}
\iomode{stdin} % Hint: stdin / interactive
% HINT: Pola konkurs, etap, day, date uzupelnia kierownik konkursu.
\konkurs{SKN 2022/23}
\day{Dzień 1}
\date{}
\group{A} % HINT: Sama literka
\Time{10} % HINT: Nie dodawać "s"
\RAM{256} % HINT: Nie dodawać "MB"

\history{01.01.2000}{usr, utworzono zadnaie}{1.0} 
% uzupełnić, w przypadku zmian dodać nową linijkę, nie usuwać starej

\start{}
\ \ Lorem ipsum dolor ssist amdet, consectetur adipiscing elit.
Donec rhoncus torstor vitae lorem luctus maximus.
Donec bibendum, dui eget diganissim viverra, lectus leo scelerisque metus, ut sagittis ligula nisl ut purus.
Etiam imperdiet luctus leo, et molestie diam congue auctor. Sed sapien eros, imperdiet in fermentum $\tc{a}$, faucibus ut ex cośtam. 
%pusta linijka, by zacząć akapit od wcięcia

Lorem ipsum dolor sit amet, consectetur adipiscing elit.
Donec rhoncus tortor vitae lorem luctus maximus.

\tSection{Wejście}

W pierwszym i jedynym wierszu wejścia znajduje się jedna liczba $\tc{n}$, ($1\leq \tc{n}\leq 1\ 000\ 000$)

\tSection{Wyjście}

Lorem ipsum dolor ssist amdet, consectetur adipiscing elit.
Donec rhoncus torstor vitae lorem luctus maximus.
Donec bibendum, dui eget diganissim viverra, lectus leo scelerisque metus, ut sagittis ligula nisl ut purus.

W pierwszym i jedynym wierszu wyjścia powinna znajdować się jedna liczba $z$, wynik zadania.

\tSection{Przykład}
\makecompactexample{a}  % abc0.in -> \makecompactexample{}
                        % abc0a.in -> \makecompactexample{a}
                        % abc0b.in -> \makecompactexample{b}

                        % Alternatywny sposób wypiswania przykładu                    


\makecompactexample{a}  % abc0.in -> \makecompactexample{}
                        % abc0a.in -> \makecompactexample{a}
                        % abc0b.in -> \makecompactexample{b}

                        % Alternatywny sposób wypiswania przykładu                    


\makecompactexample{a}

% \tSmallSection{Wyjaśnienie}
% X D P
% \makestandardexample{a} 
\noSkiptSection{Wyjaśnienie przykładów:}

\tSmallSection{Przykład 1}

Lorem ipsum dolor ssist amdet, consectetur adipiscing elit.
Lorem ipsum dolor ssist amdet, consectetur adipiscing elit.
Donec rhoncus torstor vitae lorem luctus maximus.
Donec bibendum, dui eget diganissim viverra, lectus leo scelerisque metus, ut sagittis ligula nisl ut purus.

\tSmallSection{Przykład 2}

Lorem ipsum dolor ssist amdet, consectetur adipiscing elit.
Lorem ipsum dolor ssist amdet, consectetur adipiscing elit.



\tSection{Testy przykładowe:}
\begin{enumerate}
  \setlength\itemindent{-13pt}
  \item[] \tc{1ocen:}
  $\tc{n} = 69$, Założenie
  \item[] \tc{2ocen:}
  $\tc{n} = 420$, założenie 2
\end{enumerate}

\tSection{Ocenianie}

Zestaw testów dzieli się na następujące podzadania.
Testy do każdego podzadania składają~się z~jednej lub~większej liczby osobnych grup testów.

  \begin{center}
  \begin{tabular}{|c|c|c|}
    \hline
      \tc{Nr} & \tc{Ograniczenia} & \tc{Punkty} \\
    \hline
      1 & $\tc{n} \leq 1000$ & 30 \\ \hline
      2 & $\tc{n} \leq 1\,000\,000$ & 70 \\ \hline
  \end{tabular}
  \end{center}

\finish{}

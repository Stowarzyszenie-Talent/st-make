\documentclass[opi]{talentTex}

\title{Przykładowe zadanie z długą nazwą}
\id{abc}
\group{}

%%%%%%%%%%%%%%%%%%
% Przydatne komendy:
% \pagebreak % komenda zaczynająca nową stronę pdfa
% ~ % słowa oddzielone '~' zamiast ' ' w~taki sposób będą zawsze koło siebie
% $text_mat$ % pozwala na pisanie wyrażeń matematycznych
% Dokumentacja:
%   https://www.overleaf.com/learn

%%%%%%%%%%%%%%%%%%
% Komendy talentowe
% \start{}  % Rozpoczyna treść, musi być na samym początku treści zadania
% \finish{} % Kończy treść, musi być na samym końcu treści zadania
% \tSection{text} % Nagłówek w stylu talentu
% \tNoSkipSection{text} % Jak wyżej, tylko bez odstępu od poprzedniego akapitu
% \tSmallSection{text} % Mały nagłówek w stylu talentu
% \tc{text} % Styl używany do oznaczania zmiennych
% \makecompactexample{id}   % 2 style dodawania automatycznie testów "zad0{id}" z paczki
% \makestandardexample{id}  % compact - obok siebie, standard - pod sobą
% Przy kompilacji testy są automatycznie czytane z folderów ./in i ./out więc najpierw je stwórz

\start{}

\section*{Sekcja}
    Uzupełnij ten plik według własnych upodobań stylistycznych.

    Jedyna ważna rzecz, to aby każdy po przeczytaniu umiał rozwiązać to zadanie.


\finish{}
